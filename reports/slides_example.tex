\newcommand*{\PathToAssets}{../docs_src}%
\newcommand*{\PathToOutput}{../_output}%
% \newcommand*{\PathToBibFile}{../../../bibliography.bib}%


\documentclass[ignorenonframetext, 9pt]{beamer}
\usepackage{my_beamer_header}
\usepackage{my_common_header}


\title[Short Title]{
Long Title
}
\author[Short Author Name]{Author Name Here}
\institute[Short Affiliation]{Long Affiliation}
\date{\today}


\begin{document}
\begin{frame}[noframenumbering, plain]
\titlepage
\end{frame}

\begin{frame}
  \frametitle{This paper} % Abstract?
  \begin{itemize}
  \item \alert{Question:} 
  What?
  \item \alert{Approach:} 
  How?
  \item \alert{Main Result:} 
  We find this.
  \item \alert{Application:}
  Our finding has implications for this.
  \end{itemize}
\end{frame}

% \begin{frame}[noframenumbering, plain]
%   \frametitle{Table of Contents}
%   \setbeamertemplate{section in toc}[sections numbered]
%   \tableofcontents%[hideallsubsections]
% \end{frame}

\begin{frame}
\frametitle{Definitions}
\begin{itemize}
\item Let this be ...
\end{itemize}
\end{frame}

\begin{frame}
  \frametitle{Related Literature}
  \begin{itemize}
  \item Initial motivation for term structure of equity lit.: value stocks (high book-to-market stocks) have higher risk adjusted expected returns than growth stocks. Does difference in cash flow growth explain premium?
  \\ \light{Lettau and Wachter (2007); Hansen, Heaton, Li (2008)}
  \item First direct measurement used options data and leverages put-call parity: put and call payoffs depend on price changes, not total return. Returns on short-term strips are higher than aggregate stock market, contrary to prediction of leading asset pricing models.
  % Hedging a call with a put, underlying, and bond still leave you exposed to dividend risk. Difference is the present value of the dividends. 
  \\ \light{Binsbergen, Brandt, Koijen (2012)}
  \item Phenomenon exists across countries and asset classes: cross-section of equity, nominal and corporate bonds (in Sharpe ratios), credit default swaps, volatility, housing.
  \\ \light{e.g., Palhares (2012), Giglio, Maggiori, Stroebel (2015); Binsbergen, Koijen (2017); Gormsen, Lazarus (2019)}
  \item Examining macroeconomic models through the lens of asset pricing. Growth expecations and dividend futures prices. 
  \\ \light{Borovi{\v{c}}ka, Hansen (2014); Gormsen, Koijen (2020)}
  \end{itemize}
\end{frame}

\section{Stylized Facts}


\begin{frame}
  \frametitle{The Term Structure of Equity}
  \begin{itemize}
  \item This is in the data...
  \item That...
  \end{itemize}
\label{slide:stylized_facts}
\end{frame}

\section{Model}


\begin{frame}{Model Setup}
\begin{itemize}
\item Consider a model in which...
\end{itemize}
\label{slide:state-space-framework}
\hyperlink{slide:stylized_facts}{\beamerbutton{Stylized Facts}}
\end{frame}

% \begin{frame}
% \centering
% \includegraphics[width=1.1\linewidth]{\PathToOutput/example_plot.png}
% \end{frame}

\begin{frame}[plain]
  \tiny
  \vspace{-0.5cm}
  \centering
  \textbf{Out-of-Sample $R^2$ Results by Dataset and Model}\\
  \vspace{0.2cm}
  % R2oos Results by Dataset and Model - tabular content only
% Generated automatically by create_results_tables2.py
\scriptsize
\setlength{\tabcolsep}{1.5pt}
\renewcommand{\arraystretch}{0.9}
\begin{tabular}{@{}lrrrrrrrrrr@{}}
\toprule
 & HistAvg & ARIMA & Theta & DeepAR & NBEATS & NHITS & DLinear & NLinear & TiDE & KAN \\
\midrule
\multicolumn{11}{l}{\textbf{Returns (Portfolios)}} \\
CDS Portfolio & -- & -4.33 & \textbf{-0.31} & -- & -- & -- & -- & -- & -- & -- \\
Corporate Portfolio & -0.00 & -0.00 & -0.05 & -0.49 & 0.03 & 0.04 & 0.03 & 0.06 & \textbf{0.06} & 0.03 \\
SPX Options Portfolios & \textbf{0.00} & -0.80 & -1.06 & -0.83 & -0.52 & -0.33 & -0.49 & -0.49 & -0.58 & -0.47 \\
Treasury Portfolio & -- & \textbf{-2.05} & -2.22 & -- & -- & -- & -- & -- & -- & -- \\
\midrule
\multicolumn{11}{l}{\textbf{Other}} \\
HKM All Factor & -0.00 & \textbf{0.26} & 0.20 & -1.30 & -0.21 & -0.25 & -0.58 & -0.68 & -0.59 & -1.06 \\
HKM Monthly Factor & 0.00 & 0.14 & \textbf{0.18} & -0.71 & -173.72 & -7.06 & -3.08 & -4.17 & -2.29 & -10.15 \\
\bottomrule
\end{tabular}
\end{frame}

\begin{frame}
  \frametitle{R2oos Performance Heatmap}
  \centering
  \includegraphics[width=0.9\textwidth]{\PathToOutput/forecasting3/r2oos_heatmap.png}
  \vspace{0.2cm}

  Out-of-sample R-squared performance (blue = better predictive power)
\end{frame}


\begin{frame}
  \frametitle{Proposition}

  \begin{proposition}
    Let \(f\) be a function whose derivative exists in every point, then \(f\) is 
    a continuous function.
  \end{proposition}
  \begin{proof}
    We have that \(f\) is a differentiable function and so it must be continuous.
  \end{proof}
  

\end{frame}

\appendix
\begin{frame}
  \centering
  \textbf{Appendix}
\end{frame}

\begin{frame}
  \frametitle{Extra Material}
  \label{slide:prop7}
\alert{Proposition 7 (cont.)}
\begin{itemize}
\item This and that
% \item \citet{fama1992cross}
\end{itemize}
\end{frame}


%% Bibliographies are finicky in beamer. This was working at one point, but now it isn't.
% Instead, I enter references inline manually (e.g., Fama 19XX), and refer to the paper for the full citation.
% \begin{frame}
% \vspace{15pt}
%   \bibliography{bibliography.bib}
% \end{frame}


\end{document}




